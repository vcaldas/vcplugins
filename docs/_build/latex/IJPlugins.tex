%% Generated by Sphinx.
\def\sphinxdocclass{report}
\documentclass[letterpaper,10pt,english]{sphinxmanual}
\ifdefined\pdfpxdimen
   \let\sphinxpxdimen\pdfpxdimen\else\newdimen\sphinxpxdimen
\fi \sphinxpxdimen=49336sp\relax

\usepackage[margin=1in,marginparwidth=0.5in]{geometry}
\usepackage[utf8]{inputenc}
\ifdefined\DeclareUnicodeCharacter
  \DeclareUnicodeCharacter{00A0}{\nobreakspace}
\fi
\usepackage{cmap}
\usepackage[T1]{fontenc}
\usepackage{amsmath,amssymb,amstext}
\usepackage{babel}
\usepackage{times}
\usepackage[Bjarne]{fncychap}
\usepackage{longtable}
\usepackage{sphinx}

\usepackage{multirow}
\usepackage{eqparbox}

% Include hyperref last.
\usepackage{hyperref}
% Fix anchor placement for figures with captions.
\usepackage{hypcap}% it must be loaded after hyperref.
% Set up styles of URL: it should be placed after hyperref.
\urlstyle{same}
\addto\captionsenglish{\renewcommand{\contentsname}{Contents:}}

\addto\captionsenglish{\renewcommand{\figurename}{Fig.\@ }}
\addto\captionsenglish{\renewcommand{\tablename}{Table }}
\addto\captionsenglish{\renewcommand{\literalblockname}{Listing }}

\addto\extrasenglish{\def\pageautorefname{page}}

\setcounter{tocdepth}{1}



\title{IJ Plugins Documentation}
\date{May 16, 2017}
\release{1.0}
\author{Victor Caldas}
\newcommand{\sphinxlogo}{}
\renewcommand{\releasename}{Release}
\makeindex

\begin{document}

\maketitle
\sphinxtableofcontents
\phantomsection\label{\detokenize{index::doc}}


In the future I will add some useful text here.


\chapter{Getting Started with Plugins}
\label{\detokenize{index:welcome-to-ij-plugins-s-documentation}}\label{\detokenize{index:getting-started-with-plugins}}
This documentation will guide you through the basic steps to run ImageJ with the plugins.


\section{Introduction}
\label{\detokenize{starting/intro::doc}}\label{\detokenize{starting/intro:introduction}}

\subsection{he State of Python (3 \& 2)}
\label{\detokenize{starting/intro:he-state-of-python-3-2}}\label{\detokenize{starting/intro:intro}}
When choosing a Python interpreter, one looming question is always present:
``Should I choose Python 2 or Python 3''? The answer is a bit more subtle than
one might think.

The basic gist of the state of things is as follows:
\begin{enumerate}
\item {} 
Most production applications today use Python 2.7.

\item {} 
Python 3 is ready for the production deployment of applications today.

\item {} 
Python 2.7 will only receive necessary security updates until 2020 {\color{red}\bfseries{}{[}\#pep373\_eol{]}\_}.

\item {} 
The brand name ``Python'' encapsulates both Python 3 and Python 2.

\end{enumerate}


\subsection{Recommendations}
\label{\detokenize{starting/intro:recommendations}}
I'll be blunt:
\begin{itemize}
\item {} 
Use Python 3 for new Python applications.

\item {} 
If you're learning Python for the first time, familiarizing yourself with Python 2.7 will be very
useful, but not more useful than learning Python 3.

\item {} 
Learn both. They are both ``Python''.

\item {} 
Software that is already built often depends on Python 2.7.

\item {} 
If you are writing a new open source Python library, it's best to write it for both Python 2 and 3
simultaneously. Only supporting Python 3 for a new library you want to be widely adopted is a
political statement and will alienate many of your users. This is not a problem — slowly, over the next three years, this will become less the case.

\end{itemize}


\subsection{So.... 3?}
\label{\detokenize{starting/intro:so-3}}\begin{itemize}
\item {} 
Basic Steps

\end{itemize}


\section{Properly Installing Python}
\label{\detokenize{starting/instalation::doc}}\label{\detokenize{starting/instalation:properly-installing-python}}
There's a good chance that you already have Python on your operating system.

If so, you do not need to install or configure anything else to use Python.
Having said that, I would strongly recommend that you install the tools and
libraries described in the guides below before you start building Python
applications for real-world use. In particular, you should always install
Setuptools, Pip, and Virtualenv — they make it much easier for you to use
other third-party Python libraries.


\chapter{Additional Notes}
\label{\detokenize{index:additional-notes}}
This part of the guide, which is mostly prose, begins with some
background information about Python, then focuses on next steps.


\section{The FAQ}
\label{\detokenize{support/faq::doc}}\label{\detokenize{support/faq:the-faq}}

\subsection{BDFL}
\label{\detokenize{support/faq:bdfl}}
Guido van Rossum, the creator of Python, is often referred to as the BDFL — the
Benevolent Dictator For Life.


\subsection{Python Software Foundation}
\label{\detokenize{support/faq:python-software-foundation}}
The mission of the Python Software Foundation is to promote, protect, and
advance the Python programming language, and to support and facilitate the
growth of a diverse and international community of Python programmers.

\href{http://www.python.org/psf/}{Learn More about the PSF}.


\section{he Community}
\label{\detokenize{support/error::doc}}\label{\detokenize{support/error:he-community}}

\subsection{BDFL}
\label{\detokenize{support/error:bdfl}}
Guido van Rossum, the creator of Python, is often referred to as the BDFL — the
Benevolent Dictator For Life.


\subsection{Python Software Foundation}
\label{\detokenize{support/error:python-software-foundation}}
The mission of the Python Software Foundation is to promote, protect, and
advance the Python programming language, and to support and facilitate the
growth of a diverse and international community of Python programmers.

\href{http://www.python.org/psf/}{Learn More about the PSF}.


\chapter{Indices and tables}
\label{\detokenize{index:indices-and-tables}}\begin{itemize}
\item {} 
\DUrole{xref,std,std-ref}{genindex}

\item {} 
\DUrole{xref,std,std-ref}{modindex}

\item {} 
\DUrole{xref,std,std-ref}{search}

\end{itemize}



\renewcommand{\indexname}{Index}
\printindex
\end{document}