%% Generated by Sphinx.
\def\sphinxdocclass{report}
\documentclass[letterpaper,10pt,english]{sphinxmanual}
\ifdefined\pdfpxdimen
   \let\sphinxpxdimen\pdfpxdimen\else\newdimen\sphinxpxdimen
\fi \sphinxpxdimen=49336sp\relax

\usepackage[margin=1in,marginparwidth=0.5in]{geometry}
\usepackage[utf8]{inputenc}
\ifdefined\DeclareUnicodeCharacter
  \DeclareUnicodeCharacter{00A0}{\nobreakspace}
\fi
\usepackage{cmap}
\usepackage[T1]{fontenc}
\usepackage{amsmath,amssymb,amstext}
\usepackage{babel}
\usepackage{times}
\usepackage[Bjarne]{fncychap}
\usepackage{longtable}
\usepackage{sphinx}

\usepackage{multirow}
\usepackage{eqparbox}

% Include hyperref last.
\usepackage{hyperref}
% Fix anchor placement for figures with captions.
\usepackage{hypcap}% it must be loaded after hyperref.
% Set up styles of URL: it should be placed after hyperref.
\urlstyle{same}
\addto\captionsenglish{\renewcommand{\contentsname}{Contents:}}

\addto\captionsenglish{\renewcommand{\figurename}{Fig.\@ }}
\addto\captionsenglish{\renewcommand{\tablename}{Table }}
\addto\captionsenglish{\renewcommand{\literalblockname}{Listing }}

\addto\extrasenglish{\def\pageautorefname{page}}

\setcounter{tocdepth}{1}



\title{IJ Plugins Documentation}
\date{May 17, 2017}
\release{1.0}
\author{Victor Caldas}
\newcommand{\sphinxlogo}{}
\renewcommand{\releasename}{Release}
\makeindex

\begin{document}

\maketitle
\sphinxtableofcontents
\phantomsection\label{\detokenize{index::doc}}


In the future I will add some useful text here.


\chapter{Getting Started with Plugins}
\label{\detokenize{index:getting-started-with-plugins}}\label{\detokenize{index:ij-plugins-s-documentation}}
This documentation will guide you through the basic steps to run ImageJ with the plugins.


\section{Properly Installing IJ Plugins}
\label{\detokenize{starting/instalation:properly-installing-ij-plugins}}\label{\detokenize{starting/instalation::doc}}
The easiest way to stay up to date with the plugins developed is to add the
repository to Fiji automatic updates. Below, you will learn how to do that.

1 - Go to \sphinxstylestrong{Help \textgreater{} Update ...} on the Menubar.

\noindent\sphinxincludegraphics{{fig1}.png}

It may take some time to fetch all updates.

2 - Inside the \sphinxstylestrong{ImageJ Updater} select \sphinxstylestrong{Manage update sites}

3 - Select \sphinxstylestrong{Add update site} and fill with the following information:
\begin{itemize}
\item {} 
Name: VuTools (you can fill with any name)

\item {} 
URL: \url{http://sites.imagej.net/Vcaldas/}

\end{itemize}

\noindent\sphinxincludegraphics{{fig2}.png}

You can close this window.

4 - A list of items to download will be displayed. You only need \sphinxstylestrong{VCplugins\_.jar}. All the others left unselected.
Hit \sphinxstylestrong{Apply changes}.

\noindent\sphinxincludegraphics{{fig3}.png}

5 - All ready to use!

\noindent\sphinxincludegraphics{{fig4}.png}


\chapter{Additional Notes}
\label{\detokenize{index:additional-notes}}
This part of the guide, which is mostly prose, begins with some
background information about Python, then focuses on next steps.


\section{The FAQ}
\label{\detokenize{support/faq:the-faq}}\label{\detokenize{support/faq::doc}}

\subsection{BDFL}
\label{\detokenize{support/faq:bdfl}}
Guido van Rossum, the creator of Python, is often referred to as the BDFL — the
Benevolent Dictator For Life.


\subsection{Python Software Foundation}
\label{\detokenize{support/faq:python-software-foundation}}
The mission of the Python Software Foundation is to promote, protect, and
advance the Python programming language, and to support and facilitate the
growth of a diverse and international community of Python programmers.

\href{http://www.python.org/psf/}{Learn More about the PSF}.


\section{he Community}
\label{\detokenize{support/error::doc}}\label{\detokenize{support/error:he-community}}

\subsection{BDFL}
\label{\detokenize{support/error:bdfl}}
Guido van Rossum, the creator of Python, is often referred to as the BDFL — the
Benevolent Dictator For Life.


\subsection{Python Software Foundation}
\label{\detokenize{support/error:python-software-foundation}}
The mission of the Python Software Foundation is to promote, protect, and
advance the Python programming language, and to support and facilitate the
growth of a diverse and international community of Python programmers.

\href{http://www.python.org/psf/}{Learn More about the PSF}.


\chapter{Indices and tables}
\label{\detokenize{index:indices-and-tables}}\begin{itemize}
\item {} 
\DUrole{xref,std,std-ref}{genindex}

\item {} 
\DUrole{xref,std,std-ref}{modindex}

\item {} 
\DUrole{xref,std,std-ref}{search}

\end{itemize}



\renewcommand{\indexname}{Index}
\printindex
\end{document}